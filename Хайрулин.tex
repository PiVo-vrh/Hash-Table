%% -*- coding: utf-8 -*-
\documentclass[12pt,a4paper]{scrartcl} 
\usepackage[utf8]{inputenc}
\usepackage[english,russian]{babel}
\usepackage{indentfirst}
\usepackage{misccorr}
\usepackage{graphicx}
\usepackage{amsmath}
\usepackage{listings}
\usepackage{xcolor}
\usepackage{float}
\lstset{
  language=C++,
  basicstyle=\ttfamily\small,
  keywordstyle=\color{blue},
  commentstyle=\color{gray},
  stringstyle=\color{green},
  numberstyle=\tiny\color{gray},
  stepnumber=1,
  numbersep=10pt,
  backgroundcolor=\color{white},
  showspaces=false,
  showstringspaces=false,
  showtabs=false,
  frame=none, 
  tabsize=2,
  captionpos=b,
  breaklines=true,
  breakatwhitespace=false,
  title=\lstname
}
\begin{document}
	\begin{titlepage}
		\begin{center}
			\large
			МИНИСТЕРСТВО НАУКИ И ВЫСШЕГО ОБРАЗОВАНИЯ РОССИЙСКОЙ ФЕДЕРАЦИИ
			
			Федеральное государственное бюджетное образовательное учреждение высшего образования
			
			\textbf{АДЫГЕЙСКИЙ ГОСУДАРСТВЕННЫЙ УНИВЕРСИТЕТ}
			\vspace{0.25cm}
			
			Инженерно-физический факультет
			
			Кафедра автоматизированных систем обработки информации и управления
			\vfill
			
			\textsc{Отчет по практике}\\[5mm]
			
			{\LARGE Программная реализация структуры данных \textit{Хеш-таблицы}}
			\bigskip
			
			2 курс, группа 2ИВТ АСОИУ
		\end{center}
		\vfill
		
		\newlength{\ML}
		\settowidth{\ML}{«\underline{\hspace{0.7cm}}» \underline{\hspace{2cm}}}
		\hfill\begin{minipage}{0.5\textwidth}
			Выполнил:\\
			\underline{\hspace{\ML}} В.\,Р.~Хайрулин\\
			«\underline{\hspace{0.7cm}}» \underline{\hspace{2cm}} 2025 г.
		\end{minipage}%
		\bigskip
		
		\hfill\begin{minipage}{0.5\textwidth}
			Руководитель:\\
			\underline{\hspace{\ML}} С.\,В.~Теплоухов\\
			«\underline{\hspace{0.7cm}}» \underline{\hspace{2cm}} 2025 г.
		\end{minipage}%
		\vfill
		
		\begin{center}
			Майкоп, 2025 г.
		\end{center}
	\end{titlepage}

    \newpage

\section{Введение}
\label{sec:intro}

\subsection{Текстовая формулировка задачи (Вариант 6)}
Реализовать консольное приложение для работы с хеш-таблицей. Программа должна обеспечивать операции вставки, поиска и удаления элементов, а также отображение содержимого таблицы.

\subsection{Теория метода}

Хеш-таблица — это структура данных, представляющая собой массив с необычной адресацией, задаваемой хеш-функцией. Среднее время поиска элемента составляет \( O(1) \), а в наихудшем случае — \( O(n) \), что зависит от количества коллизий.

Основные свойства хеш-таблицы:
\begin{itemize}
  \item Хеш-функция преобразует ключ в индекс массива, например, путем деления ключа на размер таблицы и взятия остатка.
  \item Для разрешения коллизий используется метод связывания, где элементы с одинаковым хешем хранятся в линейных списках.
  \item Хеш-функция должна быть быстрой и обеспечивать равномерное распределение ключей для минимизации коллизий.
\end{itemize}

Хеширование полезно для хранения большого диапазона значений в ограниченной памяти с быстрым доступом. Оно широко применяется в базах данных и компиляторах для таблиц идентификаторов. При вставке элемента ключ хешируется, и элемент добавляется в начало соответствующего списка. Поиск и удаление выполняются аналогично через хеш-функцию и линейный поиск в списке. Размер таблицы должен быть достаточным для минимизации коллизий.

\section{Ход работы}
\subsection{Выбор средств для разработки}

Для реализации хеш-таблицы я выбрал стек технологий, состоящий из:
\begin{itemize}
    \item \textbf{C++} — основной язык разработки, обеспечивающий высокую производительность и гибкость;
    \item \textbf{Шаблоны (templates)} — для создания универсального списка, используемого в хеш-таблице.
\end{itemize}

Целью было создать консольное приложение с интерактивным интерфейсом для управления хеш-таблицей.

\subsection{Код приложения}

В основе приложения лежит реализация хеш-таблицы с использованием пользовательского шаблонного списка, разделенного на несколько файлов. Ниже приведены фрагменты кода:

\begin{itemize}
    \item \textbf{Файл \texttt{list.h}:} Определение шаблонного класса \texttt{list}.
\begin{lstlisting}[language=C++, caption=Файл list.h]
#pragma once
#include <iostream>
using namespace std;

template<class T>
class list {
public:
	list();
	void pop_front();
	void pop_back();
	void removeAt(int index);
	void clear();
	void push_back(T data);
	void push_front(T data);
	void insert(T data, int index);
	void print();
	int Get_size();
	T& operator [](const int index);
	bool contains(T data);
	void remove_value(T data);
	~list();

private:
	template<class T>
	struct Node {
		Node* ptr_next;
		T data;
		Node(T data = T(), Node* ptr_next = nullptr) {
			this->data = data;
			this->ptr_next = ptr_next;
		}
	};
	int Size;
	Node<T>* head;
};
\end{lstlisting}

    \item \textbf{Файл \texttt{list.cpp}:} Реализация методов класса \texttt{list}.
\begin{lstlisting}[language=C++, caption=Фрагмент list.cpp]
template<class T>
void list<T>::push_front(T data) {
	Node<T>* current = new Node<T>(data);
	current->ptr_next = head;
	head = current;
	Size++;
}

template<class T>
void list<T>::print() {
	Node<T>* current = head;
	while (current) {
		cout << current->data << " ";
		current = current->ptr_next;
	}
}

template<class T>
list<T>::~list() {
	clear();
}

template class list<int>; // Явное указание шаблона
\end{lstlisting}

    \item \textbf{Файл \texttt{Hash_Table.h}:} Определение класса \texttt{Hash_Table}.
\begin{lstlisting}[language=C++, caption=Файл Hash_Table.h]
#pragma once
#include "list.h"
#include <iostream>
using namespace std;

class Hash_Table
{
private:
	static const int SIZE = 8;
	list<int> table[SIZE];

	int hash(int key) {
		return key % SIZE;
	}
public:
	void insert(int key);
	void remove(int key);
	void search(int key);
	void display();
};
\end{lstlisting}

    \item \textbf{Файл \texttt{Hash_Table.cpp}:} Реализация методов класса \texttt{Hash_Table}.
\begin{lstlisting}[language=C++, caption=Фрагмент Hash_Table.cpp]
void Hash_Table::insert(int key) {
	int index = hash(key);
	if (!table[index].contains(key)) {
		table[index].push_front(key);
		cout << "Элемент " << key << " добавлен.\n";
	}
	else {
		cout << "Элемент уже существует.\n";
	}
}

void Hash_Table::display() {
	for (int i = 0; i < SIZE; i++) {
		cout << "[" << i << "]: ";
		table[i].print();
		cout << endl;
	}
}
\end{lstlisting}

    \item \textbf{Файл \texttt{Source.cpp}:} Точка входа с интерактивным меню.
\begin{lstlisting}[language=C++, caption=Фрагмент Source.cpp]
int main() {
	setlocale(LC_ALL, "Russian");
	Hash_Table ht;
	int choice, value;

	do {
		cout << "\nМеню:\n";
		cout << "1. Вставить\n2. Удалить\n3. Поиск\n4. Печать\n0. Выход\n";
		cout << "Выбор: ";
		cin >> choice;

		switch (choice) {
		case 1:
			cout << "Введите значение: ";
			cin >> value;
			ht.insert(value);
			break;
		case 4:
			ht.display();
			break;
		case 0:
			cout << "Выход.\n";
			break;
		}
	} while (choice != 0);

	return 0;
}
\end{lstlisting}
\end{itemize}

\subsection{Ключевые фрагменты кода}

\textbf{Реализация списка (list):}
\begin{lstlisting}[language=C++, caption=Метод push_front()]
template<class T>
void list<T>::push_front(T data) {
	Node<T>* current = new Node<T>(data);
	current->ptr_next = head;
	head = current;
	Size++;
}
\end{lstlisting}

\textbf{Реализация хеш-таблицы (Hash_Table):}
\begin{lstlisting}[language=C++, caption=Метод insert()]
void Hash_Table::insert(int key) {
	int index = hash(key);
	if (!table[index].contains(key)) {
		table[index].push_front(key);
		cout << "Элемент " << key << " добавлен.\n";
	}
	else {
		cout << "Элемент уже существует.\n";
	}
}
\end{lstlisting}

\textbf{Реализация поиска (Hash_Table):}
\begin{lstlisting}[language=C++, caption=Метод search()]
void Hash_Table::search(int key) {
	int index = hash(key);
	if (table[index].contains(key)) {
		cout << "Элемент найден в бакете " << index << ".\n";
	}
	else {
		cout << "Элемент не найден.\n";
	}
}
\end{lstlisting}

\textbf{Отображение таблицы (Hash_Table):}
\begin{lstlisting}[language=C++, caption=Метод display()]
void Hash_Table::display() {
	for (int i = 0; i < SIZE; i++) {
		cout << "[" << i << "]: ";
		table[i].print();
		cout << endl;
	}
}
\end{lstlisting}

\subsection{Описание архитектуры приложения}

\begin{itemize}
  \item \texttt{list.h} и \texttt{list.cpp} — содержат реализацию шаблонного класса \texttt{list} для управления связными списками.
  \item \texttt{Hash_Table.h} и \texttt{Hash_Table.cpp} — содержат определение и реализацию класса \texttt{Hash_Table} с хеш-функцией и операциями над таблицей.
  \item \texttt{Source.cpp} — главный файл с точкой входа и интерактивным меню для управления хеш-таблицей.
\end{itemize}

\section{Скриншоты программы}
\label{sec:program-shots}

На следующих изображениях представлены примеры вывода консольного приложения:

\begin{figure}[H]
    \centering
    \includegraphics[width=0.7\linewidth]{console_output.png}
    \caption{Пример работы программы}
    \label{fig:console-output}
\end{figure}

\section{Источники}

\begin{thebibliography}{9}
\bibitem{Cormen-2009}Кормен Т.Х., Лейзерсон Ч.Е., Ривест Р.Л., Штайн К. Алгоритмы: построение и анализ. Москва: Вильямс, 2009 г.
\bibitem{Stroustrup-2013}Струстрап Б. Программирование: принципы и практика с использованием C++. Москва: Вильямс, 2013 г.
\bibitem{CppReference-2025}Документация C++ на cppreference.com. \url{https://en.cppreference.com}, 2025 г.
\end{thebibliography}

\section{Доступ к исходному коду и результатам работы}
\item Репозиторий: \texttt{\url{https://github.com/PiVo-vrh/Hash-Table\}} 
\end{document}